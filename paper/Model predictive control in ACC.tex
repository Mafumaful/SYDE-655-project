\documentclass{article}
\usepackage[english]{babel}
\usepackage[letterpaper,top=2cm,bottom=2cm,left=3cm,right=3cm,marginparwidth=1.75cm]{geometry}

\usepackage{amsmath}
\usepackage{graphicx}
\usepackage{indentfirst}
\usepackage{amsfonts}
\usepackage[colorlinks=true, allcolors=blue]{hyperref}

\usepackage{upgreek}

\graphicspath{{images/}}

% according devide the tag
\numberwithin{equation}{section}

\title{Model Predictive Control in Adaptive Cruise Control}
\author{Shuhao Bian}

\begin{document}
\maketitle

\begin{abstract}
    This paper presents a method of using MPC into Adaptive Cruise Control (ACC).
    The MPC is a powerful tool for solving the optimal control problem.
    We present a linear model of the system. % we can provide a nonlinear model of the system later
\end{abstract}

\section{Problem Discreption}
% Adaptive Cruise Control (ACC) is a driver assistance system that uses radar, lidar, 
% or other sensors to detect the distance between your vehicle and the vehicle in front of you, 
% and automatically adjusts your vehicle's speed to maintain a safe following distance. 
% The system is designed to reduce the driver's workload, improve safety, 
% and increase comfort during long drives.

% The aim of ACC is to help drivers maintain a safe distance from the vehicle in front 
% of them, especially in heavy traffic or on long trips. ACC is also intended to reduce 
% driver fatigue and improve fuel efficiency by optimizing speed and acceleration. 
% Additionally, ACC can alert drivers to potential collisions and provide automatic 
% braking in emergency situations. Overall, ACC can make driving safer and more comfortable, 
% while reducing the risk of accidents and improving fuel efficiency.~\cite{1337334}

Adaptive Cruise Control (ACC) is a driver assistance system that helps maintain
a safe distance from the vehicle in front, especially in heavy traffic or on
long trips. With the advancement of commercial radar, lidar, and camera
technologies, the popularity of ACC systems has increased.

Although Model Predictive Control (MPC)~\cite{rawlings2012postface} is a
popular method for solving the ACC problem, it still presents challenges due to
the computational load involved in solving large-scale optimization problems.
Therefore, this paper proposes a method to linearize the ACC system model and
use the powerful CasADi toolbox\cite{Andersson2018} for solving the optimal
control problem.

\section{Formulation}

\subsection{ACC system}
% inter vhicle following error
In this paper, there are three parameters that can be measured as below by the
ACC system:
\begin{itemize}
    \item $\Delta d$: Inter-vehicle distance following error, it represents the error between the
          desired inter-vehicle distance $d_r$ and actual inter-vehicle distance $d$. st. $\Delta d=d-d_r$.
          It's supposed to converge to zero if the ACC system works well.
    \item $\Delta v$: The speed following error, it represents the speed error between the
          host vehicle speed $v_h$ and target vehicle speed $v_r$. st. $\Delta v=v_h-v_r$.
          It's supposed to converge to zero.
    \item $\dot{v_h}$: the acceleration of the host vehicle. It's supposed to converge to
          be zero if target vehicle speed $v_r$ is constant. Because it's a simple linear model, we
          suppose the velocity of the target vehicle is constant.
\end{itemize}
\subsection{Vehicle Dynamics}
A vehicle dynamics model\cite{Takahama} will be used to describe the vehicle
dynamics and to build the MPC. The logitudinal dynamics of the host vehicle is
given by (\ref{eq:logitudinal_d}):

\begin{equation}
    \begin{aligned}
        m\dot{v_h}=F_f-r_{travel}\label{eq:logitudinal_d} \\
    \end{aligned}
\end{equation}

\noindent where $m$ is the mass of the host vehicle, $F_f$ is the total traction force of the host vehicle,
$r_{travel}$ is the rolling resistance of the host vehicle. The input/output relationship of
the vehicle dynamics model is given by the differential equation (\ref{eq:actuation_d}):

\begin{equation}
    \left\{\begin{array}{l}\label{eq:actuation_d}
        \dot{x}_{f}=f_{a c t}\left(x_{f}, u\right) \\
        a_{f}=h_{a c t}\left(x_{f}\right)
    \end{array}\right.
\end{equation}

\noindent where $x_f=\mathbb{R}^{n_{f}}$ is the state of the vehicle dynamics model, $u$ is the input of the
vehicle system~(\ref{eq:actuation_d}) which is often a control input calculated by embeded
micro cruise controller. $r_{travel}$ is travel resistance which can be calculated by the
following equation~(\ref{eq:travel_resistance}):

\begin{equation}
    r_{travel}=r_{air}v_h^2 + r_{roll}(v_h)+r_{accel}\dot{v_h}+r_{grad}(\theta)\label{eq:travel_resistance}
\end{equation}

\noindent where $r_{air}$ is the air resistance coefficient, $r_{roll}$ is the rolling resistance,
the $r_{accel}$ is the acceleration resistance coefficient, $r_{grad}$ is the gradient resistance,
$\theta$ is the slope angle.

\subsection{State-Space Model for ACC System}

To construct a plant model of the Adaptive Cruise Control (ACC) system, it is
necessary to consider both the desired inter-vehicle distance $d_r$ and the
minimum inter-vehicle distance $d_0$.

\begin{equation}
    d_r = T_{hw}v_h+d_0\label{eq:inter_vehicle_distance}
\end{equation}

\noindent where $T_{hw}$ is the headway time, $d_0$ is the minimum inter-vehicle distance.

In this paper, the state variables of the ACC system are as $x=[\Delta d,\Delta
    v,x_f^T]^T$, $x\in \mathbb{R}^{2+n_f}$, the state-space model is formulated as
(\ref{eq:state_space_model}):

\begin{equation}
    \left\{\begin{array}{l}\label{eq:state_space_model}
        \dot{x}=f_{a c t}\left(x, u\right) + Gv + Hw \\
        y=Cx+Jv
    \end{array}\right.
\end{equation}

\noindent where
$f_{a c t}(x,u)=\left[\begin{array}{c}
            \Delta v-T_{h w} x_{f}       \\
            -h_{a c t}\left(x_{f}\right) \\
            f_{a c t}\left(x_{f}, u\right)
        \end{array}\right], \quad G=\left[\begin{array}{c}
            T_{h w} / m \\
            1 / m       \\
            0
        \end{array}\right], \quad H=\left[\begin{array}{l}
            0 \\
            1 \\
            0
        \end{array}\right],C=\left[\begin{array}{lll}
            1 & 0 & 0 \\
            0 & 1 & 0 \\
            0 & 0 & 1
        \end{array}\right], \quad J=\left[\begin{array}{c}
            0 \\
            0 \\
            -1 / m
        \end{array}\right]$.

\noindent where $u \in \mathbb{R}$ is the input of the plant and
$y = [\Delta d, \Delta v, \dot{v_h}]^T \in \mathbb{R}^{3}$ is the output of the plant.

\section{Optimal-based Control Method}

\subsection{Controller Model Design}
In order to reduce the computational cost, the ACC system model is linearized
as (\ref{eq:linearized_model}):

\begin{equation}
    \left\{\begin{array}{l}\label{eq:linearized_model}
        \dot{a_f}=A_f(t)a_f+B_f(t)u \\
        a_f=C_fx_f
    \end{array}\right.
\end{equation}

\noindent where the acceleration of the linearized model is $a_f\in \mathbb{R}$, the input of the
linearized model is $u$, the output of the linearized model is $a_f$ and $A_f(t)$, $B_f(t)$, $C_f$ are
shown as (\ref{eq:linearized_m}):

\begin{equation}
    \begin{array}{l}\label{eq:linearized_m}
        A_{f}(t)=\left\{\begin{array}{ll}
                            -\frac{1}{T_{e n g}}, & \text { if } u(t) \geq a_{t h r_{-} o f f} \\
                            -\frac{1}{T_{b r k}}, & \text { if } u(t)<a_{t h r_{-} o f f}
                        \end{array}\right.                                  \\
        B_{f}(t)=\left\{\begin{array}{ll}
                            \frac{K_{e n g}(t)}{T_{e n g}}, & \text { if } u(t) \geq a_{t h r_{-} o f f} \\
                            \frac{K_{b r k}(t)}{T_{b r k}}, & \text { if } u(t)<a_{t h r_{-} o f f}
                        \end{array}\right. \\
        C_{f}=1 .
    \end{array}
\end{equation}

\noindent where $T_{e n g}$ is the constant of acceleration of acceleration engine,
$T_{b r k}$ is time constant of deceleration using brake, $a_{t h r_{-} o f f}$ is the threshold
of acceleration, $K_{e n g}(t)$ and $K_{b r k}(t)$ are the gain of acceleration engine and brake
respectively.

In the system (\ref{eq:linearized_model}), the dynamics of the vehicle is
simplified as a first-order delay system. The system has two modes:a
acceleration mode and a deceleration mode.

So the prediction model (\ref{eq:state_space_model}) can be simplified as
(\ref{eq:prediction_m}).

\begin{equation}
    \left\{\begin{array}{l}\label{eq:prediction_m}
        \dot{x}=A(t)x+B(t)u+Gv+Hw \\
        y=Cx+Jv
    \end{array}\right.
\end{equation}

\noindent where $x \in \mathbb{R}^3$ and $A(t)=\left[\begin{array}{ccc}
            0 & 1 & -T_{h w} \\
            0 & 0 & -1       \\
            0 & 0 & A_{f}(t)
        \end{array}\right], \quad B(t)=\left[\begin{array}{c}
            0 \\
            0 \\
            B_{f}(t)
        \end{array}\right]$.
\subsection{Optimization Problem}

\section{Preliminary Results}
\section{Conclusion}

In this paper, we proposed a MPC controller for ACC system. The simulation
results show that the proposed controller can achieve the desired performance.

\bibliographystyle{unsrt}
\bibliography{citation}

\end{document}